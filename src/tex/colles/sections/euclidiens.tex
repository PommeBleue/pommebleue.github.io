\subsection{Matrices $M$ telles que $M+I_n$ est inversible}
\begin{exercice}
	On pose $\mathcal E=\lbrace M\in\mathcal M_n(\R)\ |\ -1\notin \text{Sp}(M)\rbrace$.
	\begin{enumerate}
		\item Montrer que $\mathcal O_n(\R)\cap \mathcal E=\mathcal{SO}_n(\R)\cap \mathcal E$;
		\item Montrer que si $A\in\mathcal A_n(\R)$ (ensemble des matrices antisymétriques), alors $\Sp(A)\subset i\R$;
		\item Montrer que $\theta : M \mapsto (I_n-M)(I_n+M)^{-1}$ définit une involution de $\mathcal E$;
		\item Montrer que $\theta$ induit une bijection $\tilde\theta$ de $\mathcal A_n(\R)$ sur $\mathcal{SO}_n(\R)\cap \mathcal E$.
	\end{enumerate}
\end{exercice}

\begin{correction}
	\begin{enumerate}
		\item Une matrice de $\mathcal O_n(\R)\cap \mathcal E$, par théorème de réduction des matrices orthogonales,
		est semblable à une matrice diagonale par blocs, dont les blocs diagonaux sont de la forme $R_\theta$, $I_p$ et $-I_q$.
		Comme $-1$ n'est pas dans son spectre, $q=0$ nécessairement, et alors cette matrice est de déterminent $1$ ($\det R_\theta = 1$).
		La réciproque est claire, vu l'inclusion $\mathcal{SO}_n(\R)\subset \mathcal{O}_n(\R)$.

		\item Posons $\varphi:(x,y)\in\C^n\mapsto x^T\overline y$, forme sesquilinéaire. 
		Soit $\lambda$ valeur propre complexe de $A$. 
		On a $Ax=\lambda x$ avec $x\in \C^n\backslash\lbrace 0\rbrace$. 
		On a $\varphi(\lambda x,x)=\varphi(Ax,x)=-\varphi(x,Ax)=-\varphi(x,\lambda x)$.
		Mais $\varphi(x,x)\neq 0$ car $x\neq 0$ et $\varphi(\lambda x,x)=\lambda\varphi(x,x)$ et $\varphi(x,\lambda x)=\overline\lambda\varphi(x,x)$,
		donc $\lambda=-\overline\lambda$, soit $\lambda\in i\R$.

		\item Pour $M\in\mathcal E$, on a d'abord $\theta(M)\in\mathcal E$. 
		En effet, $\det(I_n+\theta(M))=\det(I_n+M)^{-1}\det(I_n+M+I_n-M)\neq 0$.
		Puis
		\begin{align*}
			&	(I_n-[(I_n-M)(I_n+M)^{-1}])(I_n+[(I_n-M)(I_n+M)^{-1}])^{-1}\\
			& = (I_n-[(I_n-M)(I_n+M)^{-1}])(I_n+M)(I_n+M+I_n-M)^{-1}\\
			& = (I_n+M-(I_n-M))\frac12 I_n\\
			& = M
		\end{align*}

		\item Considérons la restriction de $\theta$ à $\mathcal A_n$ (d'après la question (ii), on a bien $\mathcal A_n\subseteq \mathcal E$).
		Pour $M\in\mathcal A_n$, la question précédente donne déjà $\theta(M)\in \mathcal E$, il ne reste plus qu'a montrer que $\theta(M)\in\mathcal SO_n(\R)$.
		Il s'agit de calculer 
		\[
			[(I_n-M)(I_n+M)^{-1}]^T
		\] 
		Mais $[(I_n-M)(I_n+M)^{-1}]^T=[(I_n+M)^{-1}]^T(I_n-M)^T=(I_n-M)^{-1}(I_n+M)$ 
		d'abord parce que l'on peut inverser inverse et transposée, ensuite parce que $M$ 
		est antisymétrie. Puis $(I_n-M)$ et $(I_n+M)$ commutent comme des polynômes en $M$, le calcul devient 
		\[
			(I_n-M)^{-1}(I_n-M)(I_n+M)^{-1}(I_n+M)
		\]
		qui est bien sûr égal à $I_n$. 
		À ce stade, on a $\theta(M)\in\mathcal O_n(\R)$, mais $\theta(M)\in\mathcal E$, donc $\theta(M)\in\mathcal O_n(\R)\cap\mathcal E=\mathcal SO_n(\R)\cap\mathcal E$.


	\end{enumerate}
\end{correction}

\subsection{Angeline}
\begin{exercice}
	Soit $n\geq 2$.
	\begin{enumerate}
		\item Montrer que \[\forall M\in\mathcal O_n(\R),\quad \sum_{1\leq i,j\leq n}|m_{i,j}|\leq n^{\frac32}\tag{*}\]
		\item On suppose que (*) est une égalité. Que peut-on dire sur les coefficients de $M$ ? Et de la parité de $n$ ?
		\item Déterminer une matrice, notée dans la suite $M_2$, élément de $\mathcal O_2(\R)\cap\mathcal S_2(\R)$ satisfaisant le cas d'égalité de (*) pour $n = 2$.
		\item Démontrer qu'une condition suffisante pour qu'il existe $M\in\mathcal O_n(\R)$ telle que (*) soit une égalité est que $n$ soit une puissance de $2$.
		\item Démontrer qu'une condition nécessaire pour qu'il existe $M\in\mathcal O_n(\R)$ telle que (*) soit une égalité est que $n=2$ ou $4$ divise $n$.
	\end{enumerate}
\end{exercice}

\begin{correction}
	\begin{enumerate}
		\item Soit $M\in\mathcal O_n(\R)$. 
		Notons $\norm{\cdot}_2$ la norme euclidienne usuelle sur les matrices et $b$ la forme polaire de la forme quadraitque $\norm{\cdot}_2^2$. 
		Comme $MM^T=I_n$, on obtient $\tr(MM^T)=n$ soit $\norm{M}_2=\sqrt n$. 
		Puis, si on pose $A=(a_{i,j})$ avec $a_{i,j}=1$ si $m_{i,j}\geq 0$ et $a_{i,j}=-1$ sinon.
		\[
			|b(M,A)|\leq\norm{A}_2\norm{M}_2
		\]
		Or $b(M,A)=\sum|m_{i,j}|$ et $\norm{A}_2=\sqrt{n^2}=n$. 
		On obtient le résultat voulu.

		\item On est dans le cas d'égalité de l'inégalité de Schwarz, on a alors $M=\lambda A$.
		L'égalité $(*)$ nous donne directement en fait $\lambda=\frac\varepsilon{\sqrt n}$, $\varepsilon\in\lbrace -1,1\rbrace$. 
		On doit aussi avoir $MM^T=I_n$, soit, pour $i\neq j$
		\[
			0=\frac1n[AA^T]_{i,j}=\frac1n\sum_{k=1}^na_{i,k}a_{j,k}
		\]
		On a alors une somme nulle de réels égaux à $1$ ou $-1$, le nombre de termes de la somme est nécessairement pair, donc $n$ est pair.

		\item Poser $M_2$ comme une matrice avec que des $-1$ ou $1$ en coefficients, divisée par $\sqrt2$, suffit pour avoir (*),
		il ne reste qu'à bien placer les $1$ et $-1$ pour avoir $M_2M_2^T=I_2$. 
		Donc on veut
		\[
			m_{1,1}m_{2,1}+m_{1,2}m_{2,2}=0
		\]
		La matrice suivante convient 
		\[
			M_2=\frac1{\sqrt2}\left(\begin{array}{c c} 1 & -1\\ 1 & 1\end{array}\right)
		\]
		
		\item Il suffit de montrer que $n=2^k$, $k\geq 1$, permet de construire une matrice comme dans la question précédente.
		On va construire nos matrice par récurrence sur $k$. 
		$M_2$ a déjà été construite, et pour $k\geq 1$, 
		\[
			M_{2^{k+1}}=\frac1{\sqrt2}\left(\begin{array}{c c} M_{2^k} & -M_{2^k}\\ M_{2^k} & M_{2^k}\end{array}\right)
		\]
		
		\item Soit $M\in\mathcal O_n(\R)$ vérifiant (*).
		On suppose que $n$ est différent de $2$. Alors $n\geq 3$.
		On peut donc considérer, pour $\lbrace i,j\rbrace\in\mathcal P_2(\lbrace1,2,3\rbrace)$, les ensembles
		\[
			K_{i,j}=\lbrace k \in\inl1n,\ a_{i,k}a_{j,k}=-1\rbrace	
		\]
		Comme $\sum_{k=1}^na_{i,k}a_{j,k}=0$ pour une certaine paire $\lbrace i,j\rbrace$, ces ensembles sont tous de cardinal $n/2$. 
		On va montrer qu'ils sont de cardinal pair, ce qui achèvera de montrer que $n$ est divisible par $4$.
		Montrons que 
		\[
			\forall \lbrace i,j\rbrace\neq\lbrace p,q\rbrace,\ K_{i,j}^c\cap K_{p,q}^c=\lbrace k\in\inl1n,\ a_{1,k}=a_{2,k}=a_{3,k}\rbrace\tag*{(**)}
		\]
		Soit $\lbrace i,j\rbrace\neq\lbrace p,q\rbrace$, on a d'abord nécessairement $\lbrace i,j\rbrace\cup\lbrace p,q\rbrace=\lbrace 1,2,3\rbrace$, donc, par exemple, $p\in\lbrace i,j\rbrace$ mais $q\notin \lbrace i,j\rbrace$.
		Soit $k\in K_{i,j}^c\cap K_{p,q}^c$, alors $a_{i,k}a_{j,k}=a_{p,k}a_{q,k}=1$ (les coefficients sont des produits de $1$ et $-1$, donc nécessairement égaux à $1$ ou $-1$...),
		donc $a_{i,k}=a_{j,k}$ et $a_{p,k}=a_{q,k}$, et comme $p=i,j$, on a $a_{i,k}=a_{j,k}=a_{q,k}$ soit $a_{1,k}=a_{2,k}=a_{3,k}$.
		En passant au complémentaire dans (**), on voit que les ensembles $K_{i,j}\cup K_{p,q}$, sont tous de même cardinal, mais 
		\[
			|K_{i,j}\cup K_{p,q}|=|K_{i,j}|+|K_{p,q}|-|K_{i,j}\cap K_{p,q}|=n-|K_{i,j}\cap K_{p,q}|
		\]
		et cette relation montre que les ensembles $K_{i,j}\cap K_{p,q}$ sont tous de même cardinal.
		Or, si $\lbrace i,j\rbrace\neq\lbrace p,q\rbrace$, on a 
		\[
			k\in K_{i,j}\cap K_{p,q}\implies a_{i,k}a_{j,k}=a_{p,k}a_{q,k}=-1
		\]
		et sachant que parmi les deux paires choisies, il y a un indice commun $\sigma$, en simplifiant par $a_{\sigma,k}$,
		on trouve $a_{s,k}=a_{t,k}$ (où $\lbrace s,t\rbrace$ est la dernière paire possible dans $\mathcal P_2(\lbrace 1,2,3\rbrace)$), soit 
		$a_{s,k}a_{t,k}=1$ donc $k\in K_{s,t}^c$, mais on a toujours $k\in K_{i,j}$ par exemple, d'où $k\in K_{i,j}\cap K_{p,q}\implies k\in K_{s,t}^c\cap K_{i,j}$,
		et la réciproque est en fait vraie (se vérifie facilement), d'où $K_{i,j}\cap K_{p,q}=K_{s,t}^c\cap K_{i,j}$, donc 
		\[
			|K_{i,j}|=|K_{s,t}^c\cap K_{i,j}|+|K_{s,t}\cap K_{i,j}|=\underbrace{|K_{p,q}\cap K_{i,j}|+|K_{s,t}\cap K_{i,j}|}_{=2|K_{s,t}\cap K_{i,j}|}
		\]
		Donc $n/2=K_{i,j}$ est pair soit $n$ est divisible par $4$.
	\end{enumerate}
\end{correction}

\subsection{Un TLM pour les matrices}
\begin{exercice}
	Soit $n\geq 2$.
	On définit une relation d'ordre sur $\mathcal M_n(\R)$ par 
	\[
		A\leq B\iff B-A\in\mathcal S_n^+(\R)	
	\]
	Vérifier qu'il s'agit bien d'une relation d'ordre, et montrer que toute suite de matrices $(A_p)$
	croissante et majorée pour cet ordre converge.
\end{exercice}

\begin{correction}
	La réflexivité est simple, l'antisymétrie demande d'appliquer le théorème spectral pour se rendre compte que les valeurs propres de la différence sont toutes nuls,
	et pour la transitivité, on remarque que $X^T(A-B)X\geq 0$ et $X^T(B-C)X\geq 0$ donnent $X^T(A-C)X\geq 0$ pour tout $X$.

	Voyons le plus intéressant : pourquoi une suite telle que prise dans l'énoncé converge ? 
	On est dans un espace de dimension finie, on choisit la norme qui nous plait, et celle qui nous arrange le plus ici est la norme infinie,
	pour laquelle une suite de matrices converge si et seulement si les suites des coefficients convergent. 

	On va poser $(B_p)=(M-A_p)$ pour se ramener à des matrices symétriques. 
	Pour l'odre que l'on a défini, la nouvelle suite est décroissante, et les 
	hypothèses nous donnent pour tout $p\in\N$ et pour tout $X\in\mathcal \R^n$
	\[
		X^T(B_{p+1}-B_p)X=-X^T(A_{p+1}-A_p)X\leq 0\quad\text{et}\quad X^T(B_p)X\geq 0	
	\]
	Ainsi, pour tout $X\in\mathcal \R^n$, la suite (de réels) $(X^TB_pX)_p$ est décroissante et minorée,
	elle converge. 
	Pour $p\in\N$, on pose la forme quadratique $q_p:X\in\R^n\mapsto X^TB_pX$, de forme polaire $\varphi_p$. 
	On sait 
	\[
		\forall X,Y\in\R^n,\ \frac14[q_p(X+Y)-q_p(X-Y)]=\varphi_p(X,Y)
	\]
	En prenant $(E_1,\dots,E_n)$ la base canonique de $\R^n$, on obtient 
	\[
		\forall i,j\in\inl1n,\ \frac14[q_p(E_i+E_j)-q_p(E_i-E_j)]=\varphi_p(E_i,E_j)=[B_p]_{i,j}
	\]
	Mais sachant que $q_p(E_i+E_j)$ et $q_p(E_i-E_j)$ ont une limite finie, on en déduit que $([B_p]_{i,j})$ converge pour tout $i,j$,
	soit que $(B_p)$ converge, ou encore que $(A_p)$ converge.
\end{correction}

\subsection{Convexité (1)}
\begin{exercice}
	Montrer que l'application $\varphi : S\in\mathcal S_n(\R)\mapsto \tr(\exp(S))\in\R$ est convexe.
\end{exercice}

\begin{correction}
	[À rédiger]
\end{correction}

\subsection{Convexité (2)}
\begin{exercice}
	Soit $A\in\mathcal S_n^{++}(\R)$, $b\in\R^n$.
	Posons $J(x)=\frac12\langle Ax,x\rangle - \langle b,x\rangle$ pour tout $x\in\R^n$.
	\begin{enumerate}
		\item Montrer que $J$ est strictement convexe.
		\item Montrer que $J(x)\xrightarrow[\norm{x}\to+\infty]{}+\infty$.
		\item En déduire que $J$ admet un unique minimum sur $\R^n$.
	\end{enumerate}
\end{exercice}

\begin{correction}
	[À rédiger]
\end{correction}

\subsection{Croissance de la trace de l'exponentielle}
\begin{exercice}
	Soit $n\in\N^*$.
	\begin{enumerate}
		\item Soient $U,V\in\mathcal S_n^+(\R)$. Montrer qu'il existe $\R\in\mathcal S_n^+(\R)$ tel que $R^2 = U$ puis que $\tr(UV)\geq 0$.
		\item Soient $P\in\R[X]$ et $f:\R\to\mathcal M_n(\R)$ dérivable. Montrer que $\varphi:t\in\R\mapsto\tr(P(f(t)))$ est dérivable et calculer $f'$.
		\item Soient $A,B\in\mathcal S_n(\R)$ tels que $B-A\in\mathcal S_n^+(\R)$. Montrer 
		\[
			\tr(\exp(A))\leq\tr(\exp(B))
		\]
	\end{enumerate}
\end{exercice}

\begin{correction}
	\begin{enumerate}
		\item L'existence d'une racine carrée de matrices symétriques positives est un classique certainement présent dans votre cours, que je ne vais pas refaire ici. 
		Soient $R,S$ symétriques positives telles que $R^2=U$ et $S^2=V$.
		On a alors 
		\[
			\tr(UV)=\tr(RRSS)=\tr(RSSR)=\tr(RS(RS)^T)\geq 0
		\]
		\item Comme $\tr$ est linéaire, il suffit de montrer que $t\mapsto P(f(t))$ est dérivable.
		On va vérifier la dérivabilité de $t\mapsto f(t)^k$ pour tout $k\in\N$, et on déduira que ça vaut pour tout polynôme en faisant des combinaisons linéaires.
		On pose alors $p_k:M\in\mathcal M_n(\R)\mapsto M^k$, on doit montrer que $\psi : p_k\circ f$ est dérivable. 
		L'application $p_k$ est différentiable sur tout $\mathcal M_n(\R)$ et 
		\[
			\forall M,H\in\mathcal M_n(\R),\ \d(p_k)_M(H)=\sum_{i=0}^{k-1}M^iHM^{k-1-i}
		\]
		De sorte que, d'après le théorème des fonctions composées, $\psi$ soit différentiable 
		\[
			\forall t\in\R,\ \psi'(t)=\d(p_k\circ f)_t(1)=\d(p_k)_{f(t)}(f'(t))
		\]
		et ainsi
		\[
			\forall t\in\R,\ \psi'(t)=\sum_{i=0}^{k-1}f(t)^if'(t)f(t)^{k-1-i}	
		\]
		Et finalement, la linéarité de $\tr$ donne la dérivabilité de $\varphi$ et 
		\begin{align*}
			\forall t\in\R,\ \varphi'(t)=\tr(\psi'(t))  &=\tr\left(\sum_{i=0}^{k-1}f(t)^if'(t)f(t)^{k-1-i}\right)\\
														&=\sum_{i=0}^{k-1}\tr(f(t)^if'(t)f(t)^{k-1-i})\\
														&=\sum_{i=0}^{k-1}\tr(f(t)^{k-1}f'(t))\\
														&=k\tr(f(t)^{k-1}f'(t))
		\end{align*}

		Donc en fait, pour un $P$ quelconque, $\varphi$ est dérivable et $\varphi'(t)=\tr(P'(f(t))f'(t))$ pour tout $t$.

		\item On pose pour tout $n\in\N$, $P_n=\frac{X^n}{n!}$.
		Posons $f:t\in[0,1]\mapsto A + t(B-A)$ et $\varphi_n:t\in[0,1]\mapsto\tr(P_n(f(t)))$.
		Pour tout $n\geq 1$, $\varphi_n$ est dérivable sur $[0,1]$ et $\varphi_n'(t)=\tr(P_{n-1}(f(t))(B-A))$.
		Par continuité de $\tr$, la série $\sum\varphi_n$ converge simplement vers $t\in[0,1]\mapsto\tr(\exp(f(t)))$, et,
		si on munit $\mathcal M_n(\R)$ d'une norme sous-multiplicative $\norm{\cdot}$, on a 
		\begin{align*}
			\forall n\in\N^*,\ \forall t\in[0,1],\ |\varphi_n'(t)| &\leq\norm{\tr}_{(\mathcal M_n(\R))'}	\norm{P_{n-1}(f(t)(B-A))}\\
																		&\leq\norm{\tr}_{(\mathcal M_n(\R))'}\norm{B-A}P_{n-1}(\norm{f(t)}) 
		\end{align*}

		La croissances des fonctions polynomiales réelles associées aux $P_n$ montre $P_n({\norm{f(t)}})\leq P_n(\norm{f}_{L^{\infty}([0,1],\mathcal M_n(\R))}$).
		D'où 
		\[
			\forall n\in\N^*,\ \norm{\varphi'_n}_{L^\infty([0,1],\R)}\leq \norm{\tr}_{(\mathcal M_n(\R))'}\norm{B-A}P_n(\norm{f}_{L^{\infty}([0,1],\mathcal M_n(\R))})
		\]
		Ceci montre que $\sum\varphi_n'$ converge uniformément sur $[0,1]$, et en prenant la limite point par point,
		on voit que cette sérive converge uniformément vers $t\in[0,1]\mapsto\tr(\exp(f(t))(B-A))$.

		D'après le théorème de dérivabilité des suites et séries de fonctions, on montre alors que $\sum\varphi_n$ converge uniformément vers une fonction dérivable 
		et que $(\sum\varphi_n)'=\sum\varphi_n'$, soit que $t\in[0,1]\mapsto\tr(\exp(f(t)))$ est dérivable de dérivée $t\in[0,1]\mapsto \tr(\exp(f(t))(B-A))$.

		Avec le théorème spectral, on montre que $\exp(\mathcal S_n(\R))\subseteq \mathcal S_n^{++}(\R)$, d'où $\exp(A+t(B-A))\in\mathcal S_n^{++}(\R)$ pour tout $t\in[0,1]$,
		mais comme $B-A\in\mathcal S_n^+(\R)$, on a en fait $\varphi'(t)\geq 0$ pour tout $t\in[0,1]$ et ainsi $\varphi$ est croissante sur $[0,1]$, soit 
		\[
			\tr(\exp(A))=\varphi(0)\leq\varphi(1)=\tr(\exp(B))	
		\]
	\end{enumerate}
\end{correction}