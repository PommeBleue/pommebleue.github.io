\subsection{CNS pour qu'un sous-groupe de $\C^*$ soit fermé}
\begin{exercice}
    Donner des conditions nécessaires et suiffisantes 
    sur $z\in\C$ pour que $G_z=\lbrace e^{itz},\ t\in\Z\rbrace$
    soit un sous-groupe fermé de $\C^*$.
\end{exercice}

\begin{correction}
On procède par analyse synthèse.


\textbf{Analyse :} Soit $z=a+ib\in\C$ tel que $G_z$ est un sous-groupe fermé de $\C$.
Pour $t\in\Z$, on a $e^{tz}=e^{-bt}e^{ait}$. 
Montrons par l'absurde que $b=0$.
Supposons $b\neq 0$, traîtons les deux cas possibles : 
\begin{itemize}
	\item Si $b > 0$, on prend $(t_n)\in\Z^{\N}$ telle que $t_n\longrightarrow +\infty$. Dans ce cas, $(e^{it_nz})$ est une suite d'éléments de $G_z$. Pour $n\in\N$, on a \[|e^{it_nz}|=|e^{-bt_n}e^{iat_n}|=e^{-bt_n}\longrightarrow 0\]
	Donc $e^{it_nz}\longrightarrow 0$ et, $G_z$ étant fermé, $0\in G_z$, ce qui est exclu.
	\item Si $b < 0$, on refait le même raisonnement en prenant $(t_n)\in\Z^{\N}$ tendant vers $-\infty$, et on a $0\in G_z$, ce qui est exclu.
\end{itemize}
Donc le seul cas possible est $b=0$.


Ainsi, $z\in\R$. Pour avoir plus d'intuition sur ce que peuvent être les correctionutions au problème, on peut regarder le cas où $G_z$ est fini. Si $G_z$ est fini (on écarte le cas $z=0$ qui est trivial), c'est un sous-groupe fini de $\C^*$, donc, si on note $n$ son cardinal, on a $G_z=\mathbb U_n$.

Soit alors $x\in G_z$, il existe $t\in\Z^*$ et $k\in\Z$ tels que $x=e^{itz}=e^{\frac{2ik\pi}n}$. On a alors 
\begin{align*}
	e^{i\left(tz-\frac{2k\pi}n\right)} = 0 &\implies tz - \frac{2k\pi}n\equiv 0[2\pi]\\
										   &\implies \exists l\in\Z,\ z=\frac2t\left(l+\frac{2k}n\right)\pi\\
										   &\implies z\in\pi\Q
\end{align*}
On peut aussi remarquer que $z\in\pi\Q$ suffit pour que $G_z$ soit fini. 


\textbf{Synthèse :} Soit $z\in\R$. Si $z\in\pi\Q$, $G_z$ est fini, donc c'est un compact de $\C$ comme partie finie de $\C$, donc est un fermé de $\C$. C'est aussi un sous-groupe de $\C^*$, donc $z$ convient.


Sinon, $z\notin\pi\Q$. Montrons que dans ce cas $G_z$ n'est pas un fermé de $\C$. En effet, si $G_z$ est fermé, on peut montrer que $G_z=\mathbb U$. D'abord, $z\Z+2\pi\Z$ est un sous-groupe dense de $\R$, puisque sinon, on aurait, pour un $\alpha\in\R$, $z\Z+2\pi\Z=\alpha\Z$, donc $z\in z\Z+2\pi\Z=\alpha\Z$ et $2\pi\in z\Z+2\pi\Z=\alpha\Z$ donc il existe $k,l\in\Z^*$ ($z$ et $2\pi$ sont non nuls) tels que $z=k\alpha$ et $2\pi=l\alpha$, d'où $z=\frac{2k\pi}l\in\pi\Q$ ce qui est exclu. La fonction $\phi:x\in\R\mapsto e^{ix}$ étant continue, on a que $\phi(z\Z+2\pi\Z)=G_z$ est dense dans $\phi(\R)=\mathbb U$. En passant à l'adhérence, on a $G_z=\mathbb U$, ce qui est exclu, parce que, par exemple, $-1\notin G_z$.
\end{correction}

\subsection{Parties de $\GL_n(\R)$ compactes, non vides et stables par produit} 

\begin{exercice}
    Soit $X$ une partie de $\GL_n(\R)$ non vide, compacte et stable par produit. Montrer que $X$ est un sous-groupe de $\GL_n(\R)$.
\end{exercice}

\begin{correction}
    Soit $A\in X$. On considère la suite d'éléments $(A^n)_{n\in\N}$. Cette suite est une suite à élements dans $X$, puisque $A\in X$ et $X$ est stable par produit. De plus, $X$ étant compacte, $(A^n)$ admet une suite extraite convergente; il existe alors $\varphi : \N\to\N$ strictement croissante, $B\in X$ telles que \[A^{\varphi(n)}\longrightarrow B\]
    Soit $p\in\N^*$. On a \[A^{-p}A^{\varphi(n)}\longrightarrow A^{-p}B\]
    Or, à partir d'un certain rang, on a $A^{-p}A^{\varphi(n)}\in X$ (il suffit d'avoir $\varphi(n)>p$), on a donc une suite d'éléments de $X$ qui converge vers $A^{-p}B$, matrice qui est alors dans $X$ puisque $X$ est fermé. Ensuite, sachant l'expression suivante pour l'inverse d'une matrice : \[A^{-1}=\frac1{\text{det}A}{}^t\text{Com}(A)\]
    on en déduit que le passage à l'inverse est continu, d'où que \[A^{-\varphi(n)}\longrightarrow B^{-1}\]
    Donc $A^{-\varphi(n)}B\longrightarrow B^{-1}B=I_n$ puis $A^{-1}A^{-\varphi(n)}B\longrightarrow A^{-1}$. Or, pour tout $n\in\N$, $A^{-1}A^{-\varphi(n)}B=A^{-\varphi(n)-1}B\in X$, donc $A^{-1}\in X$ puisque $X$ est fermé. Ainsi $X$ est stable par produit et par passage à l'inverse, c'est un sous-groupe de $\GL_n(\R)$.
\end{correction}
    

\subsection{L'ensemble des polynômes unitaires scindés est un fermé}
\begin{exercice}
On munit $\R[X]$ de la norme $\norm{\cdot}_{\infty}$ définie par $\norm{\sum_{k=0}^{+\infty}a_kX^k}=\text{max}\lbrace |a_k|,\ k\in\N\rbrace$.
\begin{enumerate}
	\item Montrer que $\mathcal U$ l'ensemble des polynômes unitaires de $\R[X]$ est fermé.
	\item Soit $Q\in\mathcal U$ non constant, on note $p=\deg Q$. Montrer que 
	\[Q\text{ est scindé sur }\R\iff \forall z\in\C,\ |Q(z)|\geq |\Im(z)|^p\]
	\item Montrer que $\mathcal S$, l'ensemble des polynômes unitaires et scindés de $\R[X]$ est un fermé.
\end{enumerate}
\end{exercice}


\begin{correction}\hfill
    \begin{enumerate}
        \item Montrons que $\R[X]\backslash\mathcal U$ est un ouvert de $(\R[X],\norm\cdot_\infty)$. 
        Soit pour cela un polynôme $P$ à coefficients réels non unitaire. 
        Si $P$ est nul, la boule ouverte centrée en $P$ et de rayon $1$ ne contient 
        que des polynômes dont tous les coefficients sont \textit{strictement} plus petits que $1$,
        donc dont, en particulier, le coefficient dominant est \textit{strictement} plus petit que $1$.
        Si $P$ est non nul, alors $\cd P\in \R\backslash\lbrace 0,1\rbrace$. 
        Soit $\varepsilon > 0$ tel que $(\cd P-\varepsilon,\cd P+\varepsilon)\subset\R\backslash\lbrace0,1\rbrace$. 
        On peut choisir $\varepsilon < \min(1,\cd P)$. 
        Si $Q\in B(P,\varepsilon)$, deux cas se présentent 
        \begin{itemize}
            \item Soit $\deg P=\deg Q$, et alors $\norm{P-Q}_\infty<\varepsilon\implies|\cd P-\cd Q|<\varepsilon$. 
            Ceci donne $\cd Q\in(\cd P-\varepsilon,\cd P+\varepsilon)\subset\R\backslash\lbrace0,1\rbrace$.
            \item Soit $\deg P\neq\deg Q$, et alors on aurait $|\cd P|<\varepsilon$ ou $|\cd Q|<\varepsilon$ selon que $\deg P>\deg Q$ ou $\deg P<\deg Q$.
            Or, ayant choisi $\varepsilon < \cd P$ en particulier, on ne peut avoir que $|\cd Q|<\varepsilon<1$.
        \end{itemize}
        \item Supposons $Q$ scindé sur $\R$. 
        Notons 
        \[
            Q=\prod_{k=1}^p(X-\lambda_k),\ \lambda_k\in\R  
        \]
        Soit $z\in\C$. 
        On a d'une part 
        \[
            |Q(z)|^2=Q(z)\overline{Q(z)}=Q(z)Q(\overline z)=\prod_{k=1}^n(\lambda_k^2-2\lambda_k\Re(z)+|z|^2)    
        \]
        On considère alors $y\in\R\mapsto y^2-2\Re(z)y+|z|^2$, une application polynomiale de degré $2$ atteignant son minimum 
        en $-(-2\Re(z))/2=\Re(z)$, et ce minimum vaut $|\Im(z)|^2$.
        D'où, pour $k=1,\dots,p$, $\lambda_k^2-2\lambda_k\Re(z)+|z|^2\geq |\Im(z)|^2$, et donc finalement 
        \[
            |Q(z)|^2\geq\prod_{k=1}^p|\Im(z)|^2\geq|\Im(z)|^{2p}
        \]
        Réciproquement, si pour tout $z\in \C$, $|Q(z)|\geq|\Im(z)|^p$, alors pour toute racine $\alpha\in\C$ de $Q$,
        on a $0=|Q(\alpha)|\geq|\Im(\alpha)|^p$, soit $\Im(\alpha)=0$ et $\alpha\in\R$.

        \item Soit $(Q_n)\in\mathcal S^{\N}$, et supposons que cette suite converge vers $P\in\R[X]$.
        Soit $p=\deg P$. 
        On montre que l'on peut extraire de $(Q_n)$ une suite dont tous les termes sont des polynômes de degré égal à $p$.
        Pour cela, on montre 
        \[
            \forall q\in\N, \exists n\geq < q,\ \deg Q_n=p    
        \]
        en procédant par l'absurde : on suppose alors 
        \[
            \exists <\in\N, \forall n\geq q,\ \deg Q_n\neq p
        \]
        On a soit une infinté de $n$ tels que $\deg Q_n > p$ ou une infinité de $n$ tels que $\deg Q_n < p$. 
        Le deuxième cas n'est en fait pas possible (on laisse au lecteur les soins de justifier cela). 
        Il existe alors $\varphi : \N\to\N$ strictement croissante telle que 
        \[
            \forall n\in\N,\ \deg Q_{\varphi(n)}\geq p + 1
        \]
        Mais on a encore $\norm{Q_n-P}_\infty\to 0$, ce qui donne 
        \[
            \forall n\in\N,\ 1=|\cd Q_{\varphi(n)}|=|\cd(Q_{\varphi(n)}-P)|\leq \norm{Q_{\varphi(n)-P}}_\infty
        \]
        En passant à la limite, on a $1\leq 0$, ce qui n'est pas. 

        En conclusion, on dispose d'une extractrice $\varphi:\N\to\N$, telle que pour tout $n\in\N$, $\deg Q_{\varphi(n)}=p$.
        On a alors 
        \[
            \forall n\in\N,\forall z\in\C,\ |Q_{\varphi(n)}(z)|\geq|\Im(z)|^p    \tag*{(*)}
        \]
        Or, en notant $Q_{\varphi(n)}=\sum_{k=0}^pa_k^{(n)}X^k$ et $P=\sum_{k=0}^na_kX^k$, on a pour tout $k$, $a_k^{(n)}\to a_k$, donc pour tout $z\in\C$, $Q_n(z)\to P(z)$,
        soit, en passant à la limite dans (*)
        \[
            \forall z\in\C,\ |P(z)|\geq|\Im(z)|^p
        \]
        et ceci assure que $P$ est scindé sur $\R$. 
        De plus, la première question assure que $P$ est unitaire.
        
        Ceci achève de montrer que $\mathcal S$ est un fermé de $\R[X]$ muni de $\norm{\cdot}_\infty$.
    \end{enumerate}
\end{correction}


\subsection{Somme d'une partie fermée et d'une partie compacte}
\begin{exercice}
	Soient $E$ un espace vectoriel normé, $F$ une partie fermée de $E$ et $K$ une partie compacte de $E$. Montrer que $F+K$ est une partie fermée de $E$.
\end{exercice}

\begin{correction}
	Il suffit de l'écrire. Soit $(x_n)$ une suite d'éléments de $F+K$ qui converge vers $x\in E$. On a \[\forall n\in\N,\quad x_n=y_n+z_n,\quad y_n\in F,z_n\in K\]
	$(z_n)$ est une suite d'éléments de $K$ compact, donc il existe $\varphi :\N\to \N$ strictement croissante et $z\in K$ tels que $z_{\varphi(n)}\longrightarrow z$. Or, on a également \[x_{\varphi(n)}\longrightarrow x\]
	Mais pour tout $n\in\N$, on a $y_{\varphi(n)}=x_{\varphi(n)}-z_{\varphi(n)}\longrightarrow x-z$. $F$ étant fermé, il vient que $x-z\in F$, d'où $x\in F+K$.
\end{correction}

\subsection{Topologie du groupe orthogonal}
\begin{exercice}
	Soit $n\in\N^*$. On rappele que le groupe orthogonal est défini par \[O_n(\R):=\lbrace M\in\mathcal M_n(\R)\ |\ {}^tMM=I_n\rbrace\]
	Cet ensemble est-il fermé dans $\mathcal M_n(\R)$ ? Est-il connexe par arcs ?
\end{exercice}

\begin{correction}
	Il s'agit bien d'une partie fermée de $\mathcal M_n(\R)$. En effet, on considère l'application $\varphi : M\in\mathcal M_n(\R)\mapsto {}^tMM$. Cette application est continue car la transposition est continue (linéaire et dimension de l'espace de départ est finie) et la multiplication également (peut être justifié de la même manière...), puis $O_n(\R)=\varphi^{-1}(\lbrace I_n\rbrace)$, et $\lbrace I_n\rbrace$ est un fermé, donc le groupe orthogonal est image réciproque d'un fermé par une fonction continue, et est donc fermé.\\
	Pour la connexité par arcs, se référer au cours sur les espaces euclidiens.
\end{correction}

\subsection{Fonctions injectives de $[0,1]^2$ dans $\R$}
\begin{exercice}
    Existe-t-il des fonctions continues injectives de $[0,1]^2$ dans $\R$ ?
\end{exercice}

\begin{correction}
    La réponse est non.
    Si $c\in[0,1]^2$, alors $[0,1]^2\backslash\lbrace c\rbrace$ est encore un connexe,
    son image par $f$ est alors un connexe de $\R$, donc un intervalle. 
    Si $c$ est pris comme antécédent d'un élément dans un intervalle du type $(a,b)$ tel que $[a,b]\subseteq f([0,1]^2)$,
    alors $f([0,1]^2\backslash\lbrace c\rbrace)$ contient toujours $a$ et $b$, donc tout l'intervalle $[a,b]$, et donc on trouve deux antécédents au même élément,
    ce qui est exclu car $f$ est injective.
\end{correction}