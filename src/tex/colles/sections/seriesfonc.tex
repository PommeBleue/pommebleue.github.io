\subsection{Un exercice classique}
\begin{exercice}
	Soit $(f_n)_n$ une suite de fonctions continues convergeant uniformément sur $\R$ vers $f$. Soit $(x_n)_{n\in\N}$ une suite de réels convergeant vers $x$.\\
	Montrer que la suite $(f_n(x_n))_{n\in\N}$ converge et calculer sa limite.
\end{exercice}

\begin{correction}
	$f$ étant limite uniforme d'une suite de fonctions continues, elle est continue. Il vient alors que \[f(x_n)\xrightarrow[n\to+\infty]{} f(x)\tag{1}\]
	Puis, pour $n\in\N$, on a \[|f_n(x_n)-f(x)|=|f_n(x_n)-f(x_n)+f(x_n)-f(x)|\leq|f_n(x_n)-f(x_n)|+|f(x_n)-f(x)|\]
	Or, puisqu'il y a convergence uniforme, $|f_n(x_n)-f(x_n)|\leq \norm{f_n-f}_\infty^{\R}\xrightarrow[n\to+\infty]{} 0$. Puis (1) donne que $|f(x_n)-f(x)|\xrightarrow[n\to+\infty]{}0$. Par encardement, on a $|f_n(x_n)-f(x)|\xrightarrow[n\to+\infty]{} 0$, soit \[\boxed{f_n(x_n)\xrightarrow[n\to+\infty]{}f(x)}\]
\end{correction}

\subsection{Une question ouverte}
\begin{exercice}
	Existe-t-il une suite de polynômes convergeant uniformément sur $\R$ vers $\exp$ ?
\end{exercice}

\begin{correction}
	Soit $(p_n)_{n\in\N}$ une suite de polynômes convergeant uniformément vers $\exp$ sur $\R$, soit \[\forall \varepsilon > 0,\ \exists n_0\geq 0,\ \forall n\geq n_0,\ \forall x\in\R,\ |p_n(x)-\exp(x)|\leq \varepsilon\]
	Ainsi, si on "applique" cette phrase pour $\varepsilon = 333$, on a l'existence de $n_0\geq 0$ tel que, pour $n\geq n_0$ et pour $x\in\R$, on ait $|p_n(x)-\exp(x)|\leq 333$. Mézalor \[|p_n(x)-p_{n_0}(x)|\leq |p_n(x)-\exp(x)|+|p_{n_0}(x)-\exp(x)|\leq 666\]
	Ceci valant pour tout $n\geq n_0$ et pour tout $x\in\R$, on a que, pour tout $n\geq n_0$, le polynôme $p_n-p_{n_0}$ est borné sur $\R$, donc constant, d'où l'existence d'une suite de réels $(\alpha_n)$ telle que \[\forall n\geq n_0,\ \forall x\in\R,\ p_n(x)-p_{n_0}(x)=\alpha_n\]
	La suite $(p_n(42))_{n\in\N}$ convergeant, puisque CU implique CS, on a que la suite $(\alpha_n)$ converge également, il suiffit d'évaluer la précédente expression en $42$; on note $\alpha$ sa limite. Pour $x\in\R$, on a \[\forall n\geq n_0,\ p_n(x)-p_{n_0}(x)=\alpha_n\]
	On passe a la limite quand $n$ tend vers $+\infty$ et on obtient \[\exp(x)=p_{n_0}(x)+\alpha\]
	autrement dit, $\exp$ est un polynôme, ce qui est exclu.
\end{correction}

\subsection{Un exemple simple}
\begin{exercice}
	On considère la fonction définie par \[f(x):=\sum_{n=0}^{+\infty}\frac{n^x}{x^n}\]
	Déterminer le domaine $D$ de définition de $f$ et étudier la continuité de $f$ sur $D$.
\end{exercice}

\begin{correction}
	Pour $x<-1$, $n^x\to 0$ lorsque $n\to+\infty$, donc $|n^x/x^n|=o(1/|x|^n)$, par comparaison de séries à termes positifs, $\sum\frac{n^x}{x^n}$ CA donc CV. 
    Ailleurs, la série diverge grossièrement. 
    On a alors $D=(-\infty,-1)$.
    On vérifie aisément que la série de fonctions définissant $f$ converge uniformément sur tout compact de $D$, 
    ainsi, sur tout compact de $D$, $f$ est limite uniforme de fonctions continues; elle est alors continue sur $D$ tout entier.
\end{correction}
