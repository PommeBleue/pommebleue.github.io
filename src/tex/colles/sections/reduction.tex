\subsection{CNS valeur propre commune}

\begin{exercice}
	Soient $A,B\in\mathcal M_n(\C)$. Montrer que les propositions suivantes sont équivalentes :
	\begin{enumerate}
		\item $A$ et $B$ possèdent une valeur propre commune
		\item Il existe $M\in\mathcal M_n(\C)$ non nulle telle que $AM=MB$
		\item $\mu_A(B)\notin \GL_n(\C)$
	\end{enumerate}
\end{exercice}

\begin{correction}
	On montre les équivalences en montrant une chaîne d'implications.
	\begin{itemize}
		\item[] \underline{1) $\implies$ 2)} Supposons que $A$ et $B$ possèdent une valeur commune. Notons $\lambda$ cette valeur propre. D'abrod, il existe $X\in\mathcal M_{n,1}(\C)$ une colonne non nulle telle que \[AX=\lambda X\]
		Mais $\lambda$ étant valeur propre de $B$, elle est encore valeur propre de ${}^tB$, d'où l'existence de $Y\in\mathcal M_{n,1}(\C)$ une colonne non nulle telle que \[{}^tBY=\lambda Y\]
		On pose $M=X{}^tY$. On vérifie facilement que $AM=MB$, puis $M$ est non nulle puisque de rang $1$.\\


		\textit{Cette implication était la plus difficile à montrer, retenir l'idée.}
		\item[]\underline{2) $\implies$ 3)} Déjà, on a \[A^2M=A(AM)=(AM)B=MB^2\]
		puis, on vérifie facilement par récurrence que \[\forall k\in\N,\ A^kM=MB^k\]
		D'où, pour tout polynôme $P\in\C[X]$, l'égalité \[P(A)M=MP(B)\]
		En particulier, on a \[M\mu_A(B)=\mu_A(A)M=0\] \nomenclature{$\mu_A,\mu_f$}{Polynôme minimal de la matrice $A$ ou de l'endomorphisme $f$}
		Ainsi, $M$ étant non nulle, $\mu_A(B)$ est soit nulle, soit un diviseur de $0$, donc est non inversible.
		\item[]\underline{3) $\implies$ 1)} Par contraposée. Supposons que $A$ et $B$ n'ont pas de valeurs propres communes. Notons $\Sp(A)=\lbrace \lambda_1,\dots,\lambda_p\rbrace$ le spectre de $A$. On écrit \[\mu_A=(X-\lambda_1)^{\alpha_1}\dots(X-\lambda_p)^{\alpha_p}\]
		Or, pour tout $i\in\inl1p$, $\lambda_i$ n'est pas valeur propre de $B$ et donc $B-\lambda_iI_n\in GL_n(\C)$. Ainsi $\mu_A(B)$ est produit de matrices inversibles et est donc inversible.
	\end{itemize}	
\end{correction}

\subsection{P(A) diagonalisable et P'(A) inversible $\implies$ A diagonalisable}
\begin{exercice}
Soit $A\in\mathcal M_n(\C)$ et $P\in\C[X]$ tel que $P(A)$ est diagonalisable et $P'(A)$ est inversible. Montrer que $A$ est diagonisable.
\end{exercice}

\begin{correction}
	On note $B=P(T)$ et $\mu_B$ le polynôme minimal de $B$. Il vient que $\mu_B\circ P$ annule $A$ et donc il existe $Q\in\C[X]$ tel que $\mu_B\circ P=Q\mu_A$. Dérivons cette égalité : \[(\mu_B'\circ P)P'=Q\mu_A'+Q'\mu_A\tag{*}\]
	On note $\text{Sp}(A)=\lbrace \lambda_1,\dots,\lambda_p\rbrace$ le spectre de $A$. Trigonalisons $A$ (on est dans $\mathcal M_n(\C)$, on peut le faire) : il existe $U\in GL_n(\C)$ tel que $A=UTU^{-1}$ avec 
	\[
		T = \left( \begin{array}{c c c} \lambda_1 & \times & \times\\ & \ddots & \times \\ & & \lambda_p\end{array} \right)
	\]
	En remarquant que $P(T)$ et $P(A)$ ont même polynôme caractéristique (en effet, $P(A)=UP(T)U^{-1}$), on arrive à en conclure que les valeurs propres de $P(A)$ sont les $P(\lambda_i)$ avec $i\in\inl1p$. Soit $\lambda\in\text{Sp}(A)$. Ayant la relation (*) et le fait que $\lambda$ est racine de $\mu_A$, on en déduit \[\mu_B'(P(\lambda))P'(\lambda)=Q(\lambda)\mu_A'(\lambda)\]
	Or $B$ est diagonalisable, donc $\mu_B'(P(\lambda))\neq 0$ ($\mu_B$ SRS), mais $P'(A)\in GL_n(\C)$, donc $P'\wedge \mu_A=1$ et donc $P'(\lambda)\neq 0$, il vient alors que \[Q(\lambda)\mu_A'(\lambda)\neq 0 \implies \mu_A'(\lambda)\neq 0\]
	Donc $\lambda$ est racine simple de $\mu_A$. Ceci valant pour toute valeur propre $\lambda$ de $A$, et donc en fait pour toute racine $\lambda$ de $\mu_A$, on en conclut que $\mu_A$ est SRS, et donc $A$ est diagonalisable.
\end{correction}

\subsection{Diagonalisation dans $\mathcal M_n(\mathbb F_p)$}
\begin{exercice}
Soit $p$ un nombre premier, $n\in\N^*$ et $A\in\mathcal M_n(\mathbb F_p)$. Montrer que \[A\quad \text{diagonalisable}\quad \iff\quad A^p=A\]
\end{exercice}

\begin{correction}
	\begin{itemize}
		\item[$\boxed{\Rightarrow}$] Supposons $A$ diagonalisable. Il vient alors l'existence de $P\in GL_n(\mathbb F_p)$ et $D\in \mathcal D_n(\mathbb F_p)$ tels que \[A=PDP^{-1}\]
		Mais alors $A^p-A=PD^pP^{-1}-PDP^{-1}$. Or, en notant 
		\[
		D = \left(\begin{array}{c c c}
			\lambda_1 & \times &\times \\
			& \ddots & \times\\
			& & \lambda_q
		\end{array}\right)
		\]
		On a \[
		D^p=\left(\begin{array}{c c c}
			\lambda_1^p & \times&\times\\
			& \ddots & \times\\
			& & \lambda_q^p
		\end{array}\right)
		\]
		Mais si $\lambda\in\mathbb F_p$, $\lambda^p=\lambda$ (Lagrange dans un groupe quelconque ou Fermat) et donc $D^p=D$ d'où $A^p=A$.
		\item[$\boxed{\Leftarrow}$] Supposons $A^p=A$. Il vient alors que $P=X^p-X\in\mathbb F_p[X]$ est un polynôme anulateur de $A$. Puis $P = X(X^{p-1}-1)$, et, en notant $Q=X^{p-1}-1$, on a \[\forall x\in\mathbb F_p^*,\quad Q(x)=0\]
		On trouve $p-1$ racines à un polynôme de degré $p-1$, d'où, sachant que $\text{cd}Q=1$, on a \[Q=\prod_{x\in\mathbb F_p^*}(X-x)\]
		Finalement $P$ est SRS et $A$ est diagonalisable.

	\end{itemize}
\end{correction}

\subsection{Matrice semblable à son double}
\begin{exercice}
	 Soit $M\in\mathcal M_n(\C)$ telle que $M\underset{\text{sb}}{\sim}2M$. Montrer que $M$ est nilpotente.
\end{exercice}

\begin{correction}
	On trigonalise $M$. Il existe $P\in GL_n(\C)$ et $T\in \mathcal T_n(\C)$ tels que $M=PTP^{-1}$. Puis, comme $M$ est semblable à $2M$, elle-même semblable à $2T$ (il suffit de multiplier par $2$ plus haut), $T$ est semblable à $2T$. Donc $T$ et $2T$ ont même polynôme caractéristique. On note $\text{Sp}(T)=\lbrace \lambda_1,\dots,\lambda_p\rbrace$ le spectre de $T$, $\chi_T$ le polynôme caractéristique de $T$. Soit $\lambda\in \text{Sp}(T)$, alors $\chi_T(2\lambda)=\chi_{2T}(2\lambda)=0$, donc $2\lambda \in \text{Sp}(T)$. Par récurrence sur $n\in\N$, on a \[\forall \lambda \in \text{Sp}(T),\ \forall n\in\N,\ 2^n\lambda\in\text{Sp}(T)\]
	Ceci est exclu lorsqu'il existe une valeur propre non nulle de $T$. Ainsi, toutes les valeurs propres sont nulles, $T$ est de diagonale nulle (et triangulaire) et est donc nilpotente, puis $M$ lui étant semblable, elle est également nilpotente.
\end{correction}

\subsection{Comparaison de polynômes minimaux}
\begin{exercice}
 Soit $E$ un $K$ espace vectoriel, $f\in L(E)$ et $G:g\in L(E)\mapsto f\circ g$. Vérifier que $G\in L(L(E))$ et comparer (sous réserve d'existence) les polynômes minimaux de $f$ et $G$.
\end{exercice}

\begin{correction}
	On laisse au lecteur les soins de vérifier que $G$ est bien un endomorphisme. 
    Supposons que $\mu_f$ existe. 
    Soit $P\in K[X]$. 
    Remarquons que pour tout $g\in L(E)$, $P(G)(g)=P(f)\circ g$ (on peut d'abord montrer cela pour $P=X^k$ pour tout $k$ entier naturel par récurrence, et on passe ensuite à tout polynôme par des combinaisons linéaires).
    Ainsi, $P(f)=0\implies P(G)=0$. 
    Ceci montre deux choses : d'abord, $G$ admet un polynôme annulateur non nul, donc $\mu_G$ existe, et $(\mu_f)\subset (\mu_G)$, soit $\mu_G$ divise $\mu_f$. 
\end{correction}

\subsection{Coefficients du polynôme caractéristique}
\begin{exercice}
	Soit $M\in\mathcal M_n(\K)$. On appelle mineur principal d'ordre $k\in\lbrace1,\dots,n\rbrace$ le déterminant d'un $M_I=(m_{i,j})_{(i,j)\in I^2}$ avec $I\subset\lbrace 1,\dots,n\rbrace$ tel que $\Card(I)=k$. Donner une expression des coefficients de $\chi_M$ en fonction des mineurs principaux.
\end{exercice}

\begin{correction}
    Correction trouvable sur \href{https://perso.eleves.ens-rennes.fr/people/amar.ahmane}{ma page personnelle}.
\end{correction}

\subsection{Limite d'une suite de matrices}
\begin{exercice}
	Soit $M\in\mathcal M_n(\K)$, $\K = \R$ ou $\C$. Donner une condition nécessaire et suffisante pour que la suite $(M^p)_{p\in\N}$ converge. Que dire sur la valeur de la limite en cas de convergence ?

	Soit $M\in\mathcal M_n(\K)$, $\K = \R$ ou $\C$. On suppose que $(M^p)_{p\in\N}$ converge. Que dire alors sur les valeurs propres complexes de $M$ ? 
	Si de plus $1$ figure parmi les valeurs propres de $M$, donner la valeur de la multiplicité de $1$ en tant que racine du polynôme minimal de $M$.
\end{exercice}

\begin{correction}
    On procède par analyse synthèse :

    \textbf{Analyse :} Supposons que la suite $(M^p)$ converge, notons $L$ sa limite.
    Remarquons qu'en écrivant $(M^p)^2=M^{2p}$ pour tout $p$ et en passant à la limite, on obtient $L^2=L$,
    $L$ est alors la matrice d'une projection. Remarquons de plus qu'en écrivant $MM^p=M^pM$ pour tout $p$ et en passant à la limite, on obtient $ML=LM$.
    Ainsi, comme $M$ et $L$ commutent, il existe $P\in\GL_n(\C)$ et $T_L$ et $T_M$ des matrices trigonales supérieures de $\mathcal M_n(\C)$ telles que 
    $M=PT_MP^{-1}$ et $L=PT_LP^{-1}$. 
    Comme $M^p\to L$, on a en fait $T_M^p\to T_L$. 
    Ainsi pour tout $i,j=1,\dots,n$, $[T_M^p]_{i,j}\to [T_L]_{i,j}$.
    En regardant les coefficients sur la diagonale, on a pour tout $i=1,\dots,n$, $\lambda_i^p\to [T_L]_{i,i}$
    où $\lambda_1,\dots,\lambda_n$ sont les valeurs propres complexes de $M$ dans l'ordre dans lequel elles apparaissent dans $T_M$.
    $L$ étant une matrice de projection, ses valeurs propres sont soit $1$ soit $0$, donc pour tout $i=1,\dots,n$,
    $\lambda_i^p\to 0$ ou $\lambda_i^p\to 1$.
    Soit $i\in\lbrace1,\dots,n\rbrace$.
    \begin{itemize}
        \item Si $\lambda_i^p\to 0$, alors $|\lambda_i|<1$ trivialement;
        \item Si $\lambda_i^p\to 1$, alors $|\lambda_i|=1$. 
        Montrons qu'en fait $\lambda_i=1$. 
        Soit $\theta\in\R$ tel que $e^{i\theta}=\lambda_i$.
        On a $|2\sin(p\theta/2)|=|e^{ip\theta}-1|\to 0$. 
        On procède par l'absurde et on suppose que $\theta \notin 2\pi\Z$.
        Deux cas de figure se présentent, soit $\theta$ est rationnel, 
        alors le sous-groupe $(\theta/2)\Z+2\pi\Z$ de $(\R,+)$ est dense dans $\R$
        (car sinon $(\theta/2)\Z+2\pi\Z=\alpha\Z$, avec $\alpha\in\R$, puis $\theta\in\alpha\Z$ et $\theta\neq0$ donne $\alpha$ rationnel, et $2\pi\in\alpha\Z$ et $2\pi\neq 0$ donne $\pi$ rationnel, ce qui n'est pas...).
        Il vient alors, par continuité de $\sin$, que $\sin((\theta/2)\Z)=\sin((\theta/2)\Z+2\pi\Z)$ est dense dans $\sin(\R)=[-1,1]$, et on trouve alors 
        que l'ensemble des termes de la suite $(|\sin(p\theta/2)|)_p$ est dense dans $[0,1]$, alors que cette suite converge... 
        Soit $\theta$ est irationnel, et dans ce cas soit $\theta$ est commensurable à $\pi$ (i.e on peut écrire $\theta=k\pi/q$ avec $k,q$ des entiers, $q>0$),
        et alors, comme il faut $\theta\notin2\pi\Z$, on a $\theta/2\notin\pi\Z$, et alors pour tout $p$, $\sin((4q+1)\theta/2)=\sin(2k\pi+\theta/2)=\sin(\theta/2)\neq 0$,
        et on exhibe ainsi une sous-suite de $(\sin(p\theta/2))$ qui ne tend pas vers $0$; finalement si $\theta$ n'est pas commensurable à $\pi$, 
        alors on vérifie facilement que $(\theta/2)\Z+2\pi\Z$ est dense dans $\R$, et on se retrouve alors dans un cas que l'on sait traiter.

        On obtient alors que $\theta\in 2\pi\Z$, soit que $\lambda_i=1$.
    \end{itemize}

    \textbf{Synthèse :} Soit à présent une matrice $M$ dont toutes les valeurs propres complexes sont soit égales à $1$,
    soit de module strictement plus petit que $1$.

    La décomposition de Dunford donne $D$ diagonalisable et $N$ nilpotente telles que $M=N+D$ et $ND=DN$.
    Pour tout $p\geq n$, on a 
    \[
        M^p=(N+D)^p=\sum_{k=0}^p\binom pkD^{p-k}N^k=\sum_{k=0}^{n-1}\binom pk D^{p-k}N^k   \tag*{(*)} 
    \]
    Soit $P\in\GL_n(\C)$ telle que $PDP^{-1}$ soit diagonale. 
    Dans un premier temps, supposons que $1$ n'est pas valeur propre de $M$.
    Dans ce cas, pour tout $k\in\lbrace0,n-1\rbrace$, $\binom pkD^k\to 0$ : en effet,
    ces matrices sont diagonales, donc pour $i=1,\dots,p$, $[\binom pkD^{p-k}]_{i,i}=\binom pk\lambda_i^{p-k}$ avec 
    $|\lambda_i|<1$, d'où $[\binom pkD^{p-k}]_{i,i}\to 0$ car $\binom pk\sim p^k/k!$ et $p^k=o(1/\lambda_i^{p-k})$.
    Il vient alors que $M^p\to 0$ comme somme finie de trucs qui tendent vers $0$.

    Si $M$ a $1$ pour valeur propre, on montre qu'en fait si $(M^p)$ converge, alors $1$ est racine simple de $\mu_M$.
    Réciproquement, si $M$ a $1$ pour valeur propre et qu'elle racine simple de $\mu_M$, alors $(M^p)$ converge.
	
	\textit{Je ne rédige pas cette partie, puisque ma solution utilise un résultat hors programme (Jordanisation des endomorphismes) en CPGE. 
	Si vous trouvez un moyen de montrer ça dans le cadre du programme, je suis preneur.}
	



    En conclusion, $(M^p)$ converge si et seulement si $M$ a toutes ses valeurs propres complexes de module strictement plus petit que $1$,
    ou si $M$ est diagonalisable et toutes ses valeurs propres complexes sont soit égales à $1$ soit de module strictement plus petit que $1$.
\end{correction}

