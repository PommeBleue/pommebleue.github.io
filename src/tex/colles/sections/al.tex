\subsection{Formes linéaires sur $\mathcal M_n(\R)$ annulées par des crochets de lie de matrices.}

\begin{exercice}
    Soit $\varphi : \mathcal M_n(\R)\to \R$ une forme linéaire vérifiant
    \[\forall M,M'\in \mathcal M_n(\R),\ \varphi(MM')=\varphi(M'M)\quad \text{ et }\quad \varphi(I_n)=n\]
    On pose $A=\lbrace MM'-M'M,\ M,M'\in \mathcal M_n(\R)\rbrace$. 
    \begin{enumerate}
        \item Montrer que $\text{Vect}(A)=\lbrace M\in \mathcal M_n(\R)\ |\ \text{tr}(M)=0\rbrace$.
        \item En déduire que $\varphi=\text{tr}$.
    \end{enumerate}
\end{exercice}

\begin{correction}\hfill
    \begin{enumerate}
        \item $A\subset \text{Vect}(A)$, donc par définition, $\forall i,j,k,l\in\inl{1}{n},\ E_{ij}E_{kl}-E_{kl}E_{ij}=\delta_{jk}E_{il}-\delta_{li}E_{kj}\in A$. 
    
        Il s'ensuit que
        \[\forall i,l\in\inl1n,\ i\neq l\implies E_{il}\in \text{Vect}(A)\quad \text{et}\quad \forall i\in\inl{2}{n},\ E_{ii}-E_{11}\in \text{Vect}(A)\]
        On pose $\mathcal F=(E_{ij})_{i\neq j}\cup(E_{ii}-E_{11})_{i\in\inl2n}$. Cette famille est clairement libre, et contient $n^2-1$ vecteurs de $\text{Vect}(A)$, d'où $\text{dim}\text{Vect}(A)\geq n^2-1$. Or $\text{Vect}(A)\subset \text{Ker tr}$, et $\text{tr}$ étant une forme linéaire non nulle, on a $\text{dim } \text{Ker tr}=n^2-1$, donc $\text{dim} \text{Vect}(A)\leq n^2-1$ donc $\text{dim} \text{Vect}(A)=n^2-1=\text{dim}\text{ Ker tr}$ puis $\text{Vect}(A)=\text{Ker tr}$.
        \item De même que pour $\text{tr}$, on a $\text{Vect}(A)\subset \text{Ker}\varphi$ donc $\text{Ker tr}\subset \text{Ker}\varphi$, donc il existe $\lambda\in\R$ tel que $\varphi=\lambda\text{tr}$. Mais $\varphi(I_n)=n=\text{tr}(I_n)$ donc $n=\lambda n$ puis $\lambda = 1$ (sauf si $n=0$, mais ce cas est trivial).
    \end{enumerate}
\end{correction}

\subsection{Familles libres de matrices de rang $1$ de $\mathcal M_n(\R)$}
\begin{exercice}
	Soit $(X_1,\dots,X_p)\in\R^n$ une famille libre de vecteurs de $\R^n$. Montrer que la famille $(X_1{}^tX_1,\dots,X_p{}^tX_p)$ est une famille libre de vecteurs de $\mathcal M_n(\R)$. Étudier la réciproque.
\end{exercice}

\begin{correction}
    Pour $i\in\inl1p$, on note $X_i=(x_i^{(1)},\dots,x_i^{(n)})$. On a alors, \[\forall i\in\inl1p,\ X_i{}^tX_i=(x_i^{(1)}X_i|\dots|x_i^{(n)}X_i)\]
    Soit $(\lambda_1,\dots,\lambda_p)\in\R^p$ telle que $\sum_{i=1}^p\lambda_iX_i{}^tX_i=0$. Or, $(X_1,\dots,X_p)$ formant une famille libre, aucun des vecteurs la constituant n'est nul, d'où \[\forall i\in\inl1p,\ \exists l_i\in\inl1p,\ x_i^{(l_i)}\neq 0\]
    Ainsi, pour $i\in\inl1p$, en regardant que la $l_i$ème colonne dans la somme nulle écrite plus haut, on a \[\sum_{j=1}^p\lambda_jx_j^{(l_i)}X_j=0\]
    $(X_1,\dots,X_p)$ étant libre, on a $\forall j\in\inl1p,\ \lambda_jx_j^{(l_i)}=0$, en particulier $\lambda_ix_i^{(l_i)}=0$ donc $\lambda_i=0$ ($x_i^{l_i}\neq 0$). Finalement, ceci étant vrai pour tout $i\in\inl1p$, on a \[\forall i\in\inl1p,\ \lambda_i=0\]
    et $(X_1{}^tX_1,\dots,X_p{}^tX_p)$ est libre dans $\mathcal M_n(\R)$ (il faut bien sûr préciser que la taille des matrices $X_i{}^tX_i$ est de $n\times n$, mais cela est bien clair).
\end{correction}

\subsection{Combinaison linéaire d'exponetielles}

\begin{exercice}
    Soient $n\geq 0$ et $x_0,\dots,x_n\in\R^*$ tels que 
    \[
        \forall i\neq j,\ (x_i-x_j)(x_i+x_j)\neq 0    
    \]
    On suppose qu'il existe des complexes $\lambda_0,\dots,\lambda_n$ et $\varepsilon > 0$ tels que 
    \[
        \forall t\in(-\varepsilon,\varepsilon),\ \sum_{k=0}^n\lambda_ke^{itx_k}\in\R    
    \]
    Montrer que pour tout $k=0,\dots,n$, on a $\lambda_k\in\R$.
\end{exercice}

\begin{correction}
    [À rédiger]
\end{correction}

\subsection{Fonctions multiplicatives de $\mathcal M_n(\K)$ dans $\K$}
\begin{exercice}
    Soit $f:\mathcal M_n(\K)\to \K$ telle que 
    \[
        \forall A,B\in\mathcal M_n(\K),\ f(AB)=f(A)f(B)    
    \]
    Montrer que $f(A)\neq 0\iff A\in\GL_n(\K)$.
\end{exercice}
    